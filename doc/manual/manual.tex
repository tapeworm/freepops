%% LyX 1.3 created this file.  For more info, see http://www.lyx.org/.
%% Do not edit unless you really know what you are doing.
\documentclass[12pt,english]{article}
\usepackage{bookman}
\usepackage[T1]{fontenc}
\usepackage[latin1]{inputenc}
\usepackage{a4wide}
\usepackage{fancyhdr}
\pagestyle{fancy}
\usepackage{array}
\usepackage{graphicx}

\makeatletter

%%%%%%%%%%%%%%%%%%%%%%%%%%%%%% LyX specific LaTeX commands.
%% Because html converters don't know tabularnewline
\providecommand{\tabularnewline}{\\}

%%%%%%%%%%%%%%%%%%%%%%%%%%%%%% User specified LaTeX commands.


\usepackage{babel}
\makeatother

% \myincludegraphics{filename}{place}{width}{caption}{label}
\newcommand{\myincludegraphicsfull}[5]{
   \begin{figure}[#2]
   \begin{center}
   \includegraphics[#3]{EPS/#1.eps}
   \caption[#4]{#5}
   \label{#1}
   \end{center}
   \end{figure}
}

%dafault usage is: \myincludegraphics{main}{!htp}{scale=0.8}

\newcommand{\myincludegraphics}[3]{
   \begin{figure}[#2]
   \begin{center}
   \includegraphics[#3]{EPS/#1.eps}
   \caption{#1}
   \label{#1}
   \end{center}
   \end{figure}
}

\newcommand{\imageref}[1]{figure \ref{#1}}




\begin{document}
\thispagestyle{empty}

\vspace*{5cm}

\begin{flushright}\huge{FreePOPs Manual}

\vspace{-6mm}

\rule{15cm}{2mm}

\vspace{2mm}

\normalsize{Written by Enrico Tassi, Nicola Cocchiaro}

\vspace{1cm}

$Date$

$Revision$

\end{flushright}

\vspace{\stretch{1}}

\hspace{\stretch{1}}Document released under the GNU/FDL license.\hspace{\stretch{1}}

\pagebreak

\tableofcontents{}

\newpage


\section{Introduction}

FreePOPs is a POP3 daemon plus a LUA interpreter and some extra libraries
for HTTP and HTML parsing. Its main purpose is translating local POP3
requests to remote HTTP actions on the supported web-mails, but it
is really more flexible: for example there is a plugin to read news
from a website as if they were mails in a mailbox. You can easily
extend FreePOPs on the fly, without even restarting it; you can add
a plugin or modify an existing one simply changing the script file
since the plugins are written in LUA and are interpreted on the fly.


\subsection{Usage situations}

FreePOPs can be useful in some situations, here we give the most obvious
ones:

\begin{itemize}
\item You are behind a firewall that closes port 110 but you need to read
your mail and the web-mail of your mail provider sucks.
\item Your mail provider does not allow you to access your mailbox with
the POP3 protocol, but only through the web-mail service.
\item You prefer looking at your mailbox instead of browsing some websites
news.
\item You have to develop a POP3 server in less than a week and you want
it to be reasonably fast and not so memory consuming.
\item You are not a C hacker, but you want to benefit from a fast POP3 server
frontend written in C and you want not to waste a month in writing
the backend in C. LUA is a really simple and tiny language, one week
is enough to learn it in a way that allows you to use it productively.
\end{itemize}

\subsection{Features}

FreePOPs is the only software I know with these features:

\begin{itemize}
\item POP3 server RFC compliant (not full featured but compliant).
\item Portable (written in C and LUA that is written in C, so everything
is written in the most portable language around the world).
\item Small (in the sense of resources usage) and reasonably fast.
\item Extremely extensible on the fly using a simple and powerful language.
\item Pretty documented.
\item Released under the GNU/GPL license (this means FreePOPs is Free Software).
\end{itemize}

\subsection{Plugins}

These are the plugins currently included in FreePOPs:

\begin{description}
\input{libero.lua.xml.b.en.xmltex}
\input{tin.lua.xml.b.en.xmltex}
\input{davmail.lua.xml.b.en.xmltex}
\input{popforward.lua.xml.b.en.xmltex}
\input{aggregator.lua.xml.b.en.xmltex}
\input{flatnuke.lua.xml.b.en.xmltex}
\input{kernel.lua.xml.b.en.xmltex}
\input{gmail.lua.xml.b.en.xmltex}
\input{yahoo.lua.xml.b.en.xmltex}
\input{squirrelmail.lua.xml.b.en.xmltex}
\input{hotmail.lua.xml.b.en.xmltex}
\input{aol.lua.xml.b.en.xmltex}
\input{netscape.lua.xml.b.en.xmltex}
\input{tre.lua.xml.b.en.xmltex}
\input{supereva.lua.xml.b.en.xmltex}
\input{mailcom.lua.xml.b.en.xmltex}
\input{mail2world.lua.xml.b.en.xmltex}
\input{juno.lua.xml.b.en.xmltex}
\end{description}

\section{History}

FreePOPs was not born from scratch. A similar project (only in the
main usage situation) is LiberoPOPs. 

The ancestor of FreePOPs is completely written in C for some uninteresting
reasons. LiberoPOPs supports {}``plugins'' but in a more static
and complex way. The POP3 server frontend could be attached to a backend
written in C, this means you have to recompile and restart LiberoPOPs
each time to change a line in a plugin. Another interesting point
is that LiberoPOPs was created from scratch in a really short time
(you have to be Italian and use a \texttt{@libero.it} mail address
to understand why), this means it was born with a lot of bugs and
FIX-ME in the code. 

The LiberoPOPs project had a quick success, because everybody needed
it, and this means we had a lot of users. In the opensource (and also
Linux) philosophy you have to release frequently and this was exactly
what we did: we used to release every two days. We were working not
with Unix users, nor hackers, but mostly with Win32 users. Suddenly
we realized that they were lazy/bored of updating the software every
2 days. The ugly Win-world has taught them that software auto-updates,
auto-install and even auto-codes probably. 

We tried to solve this pulling out of the C engine most of the change-prone
code, but this was really hard since C is not thought to do this.
After LiberoPOPs had stabilized we started to think how to solve this. 

A scripting/interpreted embedded language seemed to me a nice choice
and after a long search on the net and on the newsgroup of my university
I found LUA.. This is not the place for telling the world how good
this small language is and I won't talk more about it here. Integrating
the LUA interpreter in LiberoPOPs was not so hard and FreePOPs is
the result. Now it is really easier to write/test a plugin and (even
if it is not implemented yet) an auto-update facility is really easy
to code since there is no need to recompile the C core in most cases.


\section{FreePOPs configuration file}

FreePOPs doesn't really need a configuration. Most users shouldn't
change the configuration file. In case you are a developer or a really
curious user the configuration file is \texttt{config.lua}, placed
in the program directory under win32 or in \texttt{/etc/freepops/}
in a posix environment.

Later you will learn that plugins are associated with a mail address
domain, and some of these plugins are aliased to other domains to
make it easier to fetch some news from some sites. Read the plugin
documentation for more info about them, and maybe send as a mail with
your new alias if you want it to be integrated in the next FreePOPs
release.

Since version 0.0.11 the \texttt{config.lua} file has a policy section.
In this section you can ban or allow classes of mail addresses. This
may be useful to network administrators.


\section{FreePOPs command line parameters}

The real FreePOPs configuration is made trough command line arguments.
They are described in depth in the man page in Unix environments and
below. Keep in mind that in normal usage situations it's not necessary
to use any of these, but if you have special needs it's useful to
use the following list as reference:

\begin{description}
\item [-p~<port>,~-\hspace{1mm}-port~<port>]By default FreePOPs binds on port 2000.
To alter this behavior just use this switch.
\item [-t~<num>,~-\hspace{1mm}-threads~<num>]FreePOPs is able to manage multiple
connections, up to \emph{num}. Default is 5.
\item [-b~addr,~-\hspace{1mm}-bind~addr]Binds over \emph{addr} instead INADDR\_ANY
(0.0.0.0). \emph{addr} must be a character string containing an IPv4
network address in the dotted-quad format, {}``ddd.ddd.ddd.ddd''
or a host name.
\item [-l~logfacility,~-\hspace{1mm}-logmode~logfacility]Can be used to specify
the logging facility. \emph{logfacility} can be {}``stdout'' for
stdout (the default), {}``syslog'' to use the logging daemon or
a valid filename to log to this file.
\item [-d,~-\hspace{1mm}-daemonize]Moves the process to background releasing the tty.
\item [-P~<host>:<port>,~-\hspace{1mm}-proxy~<host>:<port>]To tell FreePOPs which
is your HTTP proxy. If \emph{port} is not set then the default 8080
is used.
\item [-A~<username>:<password>,~-\hspace{1mm}-auth~<username>:<password>]For proxies
with basic authentication, to specify username and password. Must
be used with \emph{-P} or its long form.
\item [-u~name,~-\hspace{1mm}-useragent~name]Use this useragent in http connections.
The default is {}``Firefox/0.8''. A valid example is mozilla's {}``Mozilla/5.0
(X11; U; Linux i686; en-US; rv:1.5) Gecko/20031024 Debian/1.5-2''.
\item [-s~user.group,~-\hspace{1mm}-suid~user.group]This option is used to make
freepopsd drop root privileges after binding. If you run it as a normal
user there is no need to use this option. \emph{(Not used under Windows)}
\item [-k,~-\hspace{1mm}-kill]Terminates a running FreePOPs program.
\emph{(Not used under Windows)}
\item [-x~pluginfile,~-\hspace{1mm}-toxml~pluginfile]Prints on standard
output the XML description of the plugin.
\item [-e~scriptfile~args...,~-\hspace{1mm}-execute~scriptfile~args...]
This is a full bloated LUA interpreter, the executed script has access to all
freepops libraries.  The interpreter calls the main function that must get a
table of strings and return an integer. The arguments passed to freepopsd after
the script file name are put inside the table argument.  The return value is
returned from the interpreter. 
\item [-\hspace{1mm}-fpat~authtype,~-\hspace{1mm}-force-proxy-auth-type~authtype]
To forse a specific proxy auth method.
Accepted values are: ntlm, basic, digest e gss.
\item [-\hspace{1mm}-no-icon]
To disable the win32 systray icon (windows only).
\item [-h,~-\hspace{1mm}-help]Prints the usage message.
\item [-v,~-\hspace{1mm}-verbose,~-w,~-\hspace{1mm}-veryverbose]This tells FreePOPs to log some
interesting info for bug reporting.
\end{description}
In posix environments like Debian GNU/Linux you can start FreePOPs
at boot time as a standard service. In this case the command line
switches are stored in \texttt{/etc/default/freepops}, in some rpm
based systems you should find the same file as \texttt{/etc/sysconfig/freepops}.


\section{Email client configuration}

To configure your email client you must change the pop3 server settings.
Usually you must use localhost as the pop3 host name, and 2000 as
the pop3 port. In case you install FreePOPs in another computer of
your LAN, you should use the host's name instead of localhost, while
in case you changed the default port with the \texttt{-p} switch you
will have to use that same port in your email client. You always have
to use a full email address as username, for example \texttt{something@libero.it}
instead of only \texttt{something}. This is because FreePOPs chooses
the plugin to load by looking at your username. Later we will present
all the plugins and their associated domains, and how to create an
on-the-fly binding between a mail address and a domain.


\subsection{Outlook Express tutorial}

Here's a tutorial for configuring Outlook Express in a Windows environment.
Other mail clients should be configured similarly.

\begin{itemize}
\item From the tools menu choose the \textbf{Account...} item
(see \imageref{main})
\end{itemize}
\myincludegraphics{main}{!htp}{scale=0.8}

\begin{itemize}
\item Select your account and click on \textbf{Properties} (see
\imageref{settings})
\end{itemize}
\myincludegraphics{settings}{!htp}{scale=0.8}

\begin{itemize}
\item In the Server tab type in \textbf{Incoming mail} the name of the computer
where you started FreePOPs, usually \emph{localhost}. The \textbf{Account
name} must be your complete email address, followed by the domain
name your email belongs to, for example \texttt{username@domain.com}
(see \imageref{server}).
\end{itemize}
\myincludegraphics{server}{!htp}{scale=0.8}

\begin{itemize}
\item In the \textbf{Advanced settings} tab type in \textbf{Incoming mail}
the port number, that is \emph{2000} if you accepted our settings.
\textbf{\emph{De-select}} \emph{This server requires a secure connection
(SSL)} (see \imageref{advanced}).
\end{itemize}
\myincludegraphics{advanced}{!htp}{scale=0.8}

\subsection{Proxy tutorial}

FreePOPs is able to use HTTP proxy servers. If you don't know what
they are or if there's no proxy in your local network then you may
skip this page, as the operations described herein will be useless
to you.

In order to use a HTTP proxy, FreePOPs supports the \texttt{-P} option,
or the equivalent long version \texttt{-{}-proxy}, to specify the
address and port of the proxy separated by : (colon), for example
\texttt{-P proxy.localnet.org:8080} or \texttt{-P 192.168.1.1} are
valid choices. If no port number is specified then \texttt{8080} will
be used as a default value.

If authentication is necessary in order to use a proxy, use also the
\texttt{-A username:password} option.

Remember that the values specified with the \texttt{-P} option have
precedence over any other value obtained by the operating system in
use.

In POSIX environments it's possible to use a proxy also using some
environment variables.

The environment variables that will be used are, in order of precedence,
\texttt{HTTP\_PROXY}, \texttt{http\_proxy}, \texttt{PROXY} and \texttt{proxy}. 

The current implementation supports some proxy authentication methods,
and some of them require the SSL version of FreePOPs.


\subsection{Spam/AV tutorial}

Several Windows users, in collaboration with the LiberoPOPs team
have created a tutorial for antispam and antivurus software. This
tutorial is for FreePOPs too.


\subsubsection{Norton AntiVirus, version 2002 and up}

It is necessary to have FreePOPs listen on port 110 by means of the
\texttt{-p} option and then set your email client so that it receives
mail on port 110. To change FreePOPs options, read the FAQ {}``How
do I change FreePOPs's command line switches?'' question. 


\subsubsection{Avast! AntiVirus}

In your email client, change the username like this: \texttt{email@address\#localhost:2000}
Inside the email client options, set the POP3 server port number to
110, instead of what explained in the previous tutorials. 


\subsubsection{AVG Pro 7 AntiVirus}

In your email client, modify the POP3 port number to 5300, leave unmodified
the username and server (\texttt{email@address} and \texttt{localhost}).
In AVG, enter \char`\"{}Properties > Servers > Create a POP3 mail
server (server type)\char`\"{}, in connection set Fixed host: 127.0.0.1:2000
and Local port: 5300 


\subsubsection{SpamHilator}

Configure your email client with the following parameters: POP3 server
(incoming mail): localhost POP3 server port: 110 Username: \texttt{localhost\&email@address\&2000} 


\subsubsection{Mailshield Desktop}

In Mailshield Desktop, choose \char`\"{}Edit mail account\char`\"{},
choosing the account you want to modify. In \char`\"{}Account name\char`\"{}
and \char`\"{}Email address\char`\"{} type your complete email address.
Then choose \char`\"{}Access\char`\"{}, in \char`\"{}type of Email
server\char`\"{} use \char`\"{}POP3 mail account\char`\"{}, while
in \char`\"{}incoming mail server\char`\"{} type 127.0.0.1. It may
be useful to select the option \char`\"{}Use relaxed timeouts with
this email server\char`\"{}. 


\subsubsection{K9}

Set up your email client to use 9999 as POP3 server port. Leave localhost
as server name. Then use \texttt{localhost/2000/email@address} as
username. 


\subsubsection{SpamTerminator}

Configure your email client with the following parameters: POP3 server
(incoming mail): localhost POP3 server port: 8110 Username: \texttt{email@address\#localhos}t
Then start FreePOPs with the option \texttt{-p} 110. 


\subsubsection{SpamPal}

Configure your email client with the following parameters: POP3 server
(incoming mail): localhost POP3 server port: 110 Username: \texttt{email@address@localhost:2000}


\subsection{LAN tutorial }


\subsubsection*{How to use FreePOPs as a server in a computer network (Windows-oriented
tutorial).}

A LAN is composed of 2 computers (or more, but from 2 to 100 it is
the same). We'll call \emph{Sola} the server and \emph{Cucco} the
client. FreePOPs will start on \emph{Sola} with these options: \\
\texttt{freepopsd.exe -b 0.0.0.0 -p 110} \texttt{}~\\
that means that FreePOPs will bind on 0.0.0.0 (all network interfaces,
offering the service to all) at port 110, the default POP3 port. Now
we configure the mail client on Cucco, setting the POP3 server to
\emph{Sola} and the server port to \emph{110}. Now, consider \emph{Sola}
has a monitor too and we want to read some mail from here. We have
to set the server to \emph{localhost} and the port to \emph{110}.

By default under Windows FreePOPs binds on \texttt{127.0.0.1} offering
the service only to the local computer, so the \texttt{-b} switch
is really important here.


\section{Plugins}

Here we give a detailed description of each plugin, but before starting
we explain the general way of passing special arguments to plugins
(see the specific plugin description for a detailed description of
the accepted parameters).


\subsection{Parameters}

Each plugin can receive parameters passed as an addon to username.
The following username is for the \texttt{popforward.lua} plugin:\\
\texttt{gareuselesinge@mydomain.xx?host=pop.mydomain.xx\&port=110}\\
Since you may use some antispam proxy or other program that may handle
your username and may dislike the \texttt{?} character you may use
a space instead of it.
Every following character that is not a letter or a number needs to be escaped.
So they have to be written as \texttt{\%xx}, where \texttt{xx} is the
hexadecimal code of the corresponding character (exacly as it happens in URL).
The space character could also be substituted simply by a plus characted
\texttt{+}. For example, if you have to assign the value \texttt{"My Messages"}
to the \texttt{folder} parameter, you need to write \texttt{folder=My+Messages}.Take a look to the Appendix, to find more informations about hexadecimal codes.

Another way of hacking with the username is the on-the-fly domain-plugin
binding. You may find useful to say: {}``I want to use plugin X for
domain Y without changing the config.lua file''. In this case you
have to use the plugin name (for example \texttt{popforward.lua})
as the domain name an probably you will have to pass some arguments
to the plugin using the procedure previously described. This is an
example:\\
\texttt{gareuselesinge@popforward.lua?host=pop.mymailsite.xx\&port=110}\\
Remember that in the case of the use of on-the-fly bindings there
will be no default arguments, thus \texttt{port=110} can't be omitted
as in the previous example.


\input{libero.lua.xml.en.xmltex}
\input{tin.lua.xml.en.xmltex}
\input{davmail.lua.xml.en.xmltex}
\input{popforward.lua.xml.en.xmltex}
\input{aggregator.lua.xml.en.xmltex}
This is the list of aliases for the aggregator plugin.\\
\\
\begin{tabular}{|l|l|}
\hline 
\texttt{\footnotesize aggregatordomain}&
{\footnotesize description}\tabularnewline
\hline
\hline 
\texttt{\footnotesize freepops.rss.en}&
\multicolumn{1}{l|}{\texttt{\footnotesize http://www.freepops.org/} {\footnotesize news
(English)}}\tabularnewline
\hline 
\texttt{\footnotesize freepops.rss.it}&
\texttt{\footnotesize http://www.freepops.org/} {\footnotesize news
(Italian)}\tabularnewline
\hline 
\texttt{\footnotesize flatnuke.sf.net}&
\texttt{\footnotesize http://flatnuke.sourceforge.net/} {\footnotesize news
(Italian)}\tabularnewline
\hline 
\texttt{\footnotesize ziobudda.net}&
\texttt{\footnotesize http://ziobudda.net/} {\footnotesize news (both
Italian and English)}\tabularnewline
\hline 
\texttt{\footnotesize punto-informatico.it}&
\texttt{\footnotesize http://punto-informatico.it/} {\footnotesize news
(Italian)}\tabularnewline
\hline 
\texttt{\footnotesize linuxdevices.com}&
\texttt{\footnotesize http://linuxdevices.com/} {\footnotesize news
(English)}\tabularnewline
\hline 
\texttt{\footnotesize gaim.sf.net}&
\texttt{\footnotesize http://gaim.sourceforge.net/} {\footnotesize news
(English)}\tabularnewline
\hline 
\texttt{\footnotesize securityfocus.com}&
\texttt{\footnotesize http://www.securityfocus.com/} {\footnotesize new
vulnerabilities (English)}\tabularnewline
\hline 
\texttt{\footnotesize games.gamespot.com}&
\texttt{\footnotesize http://www.gamespot.com/} {\footnotesize computer
games news (English)}\tabularnewline
\hline 
\texttt{\footnotesize news.gamespot.com}&
\texttt{\footnotesize http://www.gamespot.com/} {\footnotesize GameSpot
news (English)}\tabularnewline
\hline 
\texttt{\footnotesize kerneltrap.org}&
\texttt{\footnotesize http://kerneltrap.org} {\footnotesize news (English)}\tabularnewline
\hline 
\texttt{\footnotesize mozillaitalia.org}&
\texttt{\footnotesize http://www.mozillaitalia.org} {\footnotesize news
(Italian)}\tabularnewline
\hline 
\texttt{\footnotesize linux.kerneltrap.org}&
\texttt{\footnotesize http://linux.kerneltrap.org} {\footnotesize news
(English)}\tabularnewline
\hline
\texttt{\footnotesize linuxgazette.net}&
\texttt{\footnotesize http://linuxgazette.net} {\footnotesize news
(English)}\tabularnewline
\hline

\end{tabular}


\input{flatnuke.lua.xml.en.xmltex}
There are some alias for FlatNuke sites, see
the aggregator plugin documentation to know what this means:\\
\\
\begin{tabular}{|l|l|}
\hline 
\texttt{\footnotesize aggregatordomain}&
{\footnotesize description}\tabularnewline
\hline
\hline 
\texttt{\footnotesize freepops.en}&
\texttt{\footnotesize http://www.freepops.org/} {\footnotesize full
news (English)}\tabularnewline
\hline 
\texttt{\footnotesize freepops.it}&
\texttt{\footnotesize http://www.freepops.org/} {\footnotesize full
news (Italian)}\tabularnewline
\hline 
\texttt{\footnotesize flatnuke.it}&
\texttt{\footnotesize http://flatnuke.sourceforge.net/} {\footnotesize full
news (Italian)}\tabularnewline
\hline
\end{tabular}

\input{kernel.lua.xml.en.xmltex}
\input{gmail.lua.xml.en.xmltex}
\input{yahoo.lua.xml.en.xmltex}
\input{squirrelmail.lua.xml.en.xmltex}
\input{hotmail.lua.xml.en.xmltex}
\input{aol.lua.xml.en.xmltex}
\input{netscape.lua.xml.en.xmltex}
\input{tre.lua.xml.en.xmltex}
\input{supereva.lua.xml.en.xmltex}
\input{mailcom.lua.xml.en.xmltex}
\input{mail2world.lua.xml.en.xmltex}
\input{juno.lua.xml.en.xmltex}

\section{Creating a plugin}

Two sections follow, the first is a quick overview of what a plugin
has to do, the latter is a more detailed tutorial. Before proceeding
I suggest you read some stuff that is at the base of plugin writing:

\begin{enumerate}
\item Since plugins are written in LUA you must read at least the LUA tutorial
(HTTP://lua-users.org/wiki/LuaTutorial); many thanks to the guys who
wrote it. LUA is a quite simple scripting language, easy to learn,
and easy to read. If you are interested in this language you should
read THE book about LUA ({}``Programming in Lua'' by Roberto Ierusalimschy
HTTP://www.inf.puc-rio.br/\textasciitilde{}roberto/book/). It is a
really good book, believe me. Today I've seen that the book is completely
available online here. HTTP://www.lua.org/pil/
\item Since we have to implement a POP3 backend you should know what POP3
is. The RFC is number 1939 and is included in the doc/ directory of
the source package of FreePOPs, but you can fetch it from the net
HTTP://www.ietf.org/rfc/rfc1939.txt.
\item Read carefully this tutorial, it is hardly a good tutorial, but is
better than nothing.
\item The website contains, in the doc section, a quite good documentation
of the sources. You should keep a web browser open at the LUA modules
documentation page while writing a plugin.
\item After creating a prototype, you should read a full featured plugin.
The libero.lua plugin is really well commented, you may start there.
\item Remember that this software has an official forum (HTTP://freepops.diludovico.it)
and some authors you may ask for help.
\item FreePOPs is licensed under the GNU/GPL license, therefore any and
all software that makes use of its code must be released under the
same license. This includes plugins. For more information read the
enclosed \texttt{COPYING} file or see the license text at HTTP://www.gnu.org/licenses/gpl.html.
\end{enumerate}

\subsection{Plugins overview}

A plugin is essentially a backend for a POP3 server. The plugins are
written in LUA%
\footnote{The language website is HTTP://www.lua.org%
} while the POP3 server is written in C. Here we examine the interfaces
between The C core and the LUA plugins.


\subsubsection{The interface between the C core and a plugin}

As we explained before the C POP3 frontend has to be attached to a
LUA backend. The interface is really simple if you know the POP3 protocol.
Here we only summarize the meaning, but the RFC 1939 (included in
the \texttt{doc/} directory of the source distribution) is really
short and easy to read. As your intuition should suggest the POP3
client may ask the pop3 server to know something about the mail that
is in the mailbox and eventually retrieve/delete a message. And this
is exactly what it does.

The backend must implement all the POP3 commands (like USER, PASS,
RETR, DELE, QUIT, LIST, ...) and must give back to the frontend the
result. Let us give a simple example of a POP3 session taken from
the RFC:

\linespread{0.5}

\begin{footnotesize}

\begin{verbatim}

     1  S: <wait for connection on TCP port 110>

     2  C: <open connection>

     3  S:    +OK POP3 server 

     4  C:    USER linux@kernel.org

     5  S:    +OK now insert the password

     6  C:    PASS gpl

     7  S:    +OK linux's maildrop has 2 messages (320 octets)

     8  C:    STAT

     9  S:    +OK 1 320

    10  C:    LIST

    11  S:    +OK 2 messages (320 octets)

    12  S:    1 320

    13  S:    .

    14  C:    RETR 1

    15  S:    +OK 120 octets

    16  S:    <the POP3 server sends message 1>

    17  S:    .

    18  C:    DELE 1

    19  S:    +OK message 1 deleted

    20  C:    QUIT

    21  S:    +OK dewey POP3 server signing off (maildrop empty)

    22  C:  <close connection>

    23  S:  <wait for next connection>

\end{verbatim}

\end{footnotesize}

In this session the backend will be called for lines 4, 6, 8, 10,
14, 18, 20 (all the \texttt{C:} lines) and respectively the functions
implementing the POP3 commands will be called this way

\linespread{0.5}

\begin{footnotesize}

\begin{verbatim}

    user(p,"linux@kernel.org")

    pass(p,"gpl")

    stat(p)

    list_all(p)

    retr(p,1)

    dele(p,1)

    quit_update(p)

\end{verbatim}

\end{footnotesize}

Later I will make clear what p is. I hope we'll remove it making it
implicit for complete transparency. It is easy to understand that
there is a 1-1 mapping between POP3 commands and plugin function calls.
You can view a plugin as the implementation of the POP3 interface.


\subsubsection{The interface between a plugin and the C core}

Let us take in exam the call to \texttt{pass(p,''gpl'')}. Here the
plugin should authenticate the user (if there is a sort of authentication)
and inform the C core of the result. To achieve this each plugin function
must return an error flag, to be more accurate one of these errors:\\


\begin{tabular}{|l|p{7cm}|}
\hline 
{\footnotesize Code}&
{\footnotesize Meaning}\tabularnewline
\hline
\hline 
\texttt{\footnotesize POPSERVER\_ERR\_OK}&
{\footnotesize No error}\tabularnewline
\hline 
\texttt{\footnotesize POPSERVER\_ERR\_NETWORK}&
{\footnotesize An error concerning the network}\tabularnewline
\hline 
\texttt{\footnotesize POPSERVER\_ERR\_AUTH}&
{\footnotesize Authorization failed}\tabularnewline
\hline 
\texttt{\footnotesize POPSERVER\_ERR\_INTERNAL}&
{\footnotesize Internal error, please report the bug}\tabularnewline
\hline 
\texttt{\footnotesize POPSERVER\_ERR\_NOMSG}&
{\footnotesize The message number is out of range}\tabularnewline
\hline 
\texttt{\footnotesize POPSERVER\_ERR\_LOCKED}&
{\footnotesize Mailbox is locked by another session}\tabularnewline
\hline 
\texttt{\footnotesize POPSERVER\_ERR\_EOF}&
\multicolumn{1}{l|}{{\footnotesize End of transmission, used in the popserver\_callback}}\tabularnewline
\hline 
\texttt{\footnotesize POPSERVER\_ERR\_TOOFAST}&
{\footnotesize You are not allowed to reconnect to the server now,
wait a bit and retry}\tabularnewline
\hline 
\texttt{\footnotesize POPSERVER\_ERR\_UNKNOWN}&
{\footnotesize No idea of what error I've encountered}\tabularnewline
\hline
\end{tabular}\\
\\


In our case the most appropriate error codes are \texttt{POPSERVER\_ERR\_AUTH}
and \texttt{POPSERVER\_ERR\_OK}. This is a simple case, in which an
error code is enough. Now we analyze the more complex case of the
call to \texttt{list\_all(p)}. Here we have to return an error code
as before, but we also have to inform the C core of the size of all
messages in the mailbox. We need the p parameter passed to each plugin
function (note that that parameter may became implicit in the future).
\texttt{p} stands for the data structure that the C core expects us
to fill calling appropriate functions like \texttt{set\_mailmessage\_size(p,num,size)}
where num is the message number and size is the size in bytes. Usually
it is really common to put more functions all together. For example
when you get the message list page in a webmail you know the number
of the messages, their size and uidl so you can fill the p data structure
with all the informations for LIST, STAT, UIDL. 

The last case that we examine is \texttt{retr(p,num,data)}. Since
a mail message can be really big, there is no pretty way of downloading
the entire message without making the mail client complain about the
server death. The solution is to use a callback. Whenever the plugin
has some data to send to the client he should call the \texttt{popserver\_callback(buffer,data)}.
\texttt{data} is an opaque structure the popserver needs to accomplish
its work (note that this parameter may be removed for simplicity).
In some cases, for example if you know the message is small or you
are working on a fast network, you can fetch the whole message and
send it, but remember that this is more memory consuming.


\subsection{The art of writing a plugin (plugins tutorial)}

In this section we will write a plugin step by step, examining each
important detail. We will not write a real and complete plugin since
it may be a bit hard to follow but we will create an ad-hoc webmail
for our purposes.


\subsubsection{(step 1) The skeleton}

The first thing we will do is copy the \texttt{skeleton.lua} file
to \texttt{foo.lua} (since we will write the plugin for the \emph{foo.xx}
webmail, \emph{xx} stands for a real domain, but I don't want to mention
any websites here...). Now with your best editor (I suggest vim under
Unix and scintilla for win32, since they support syntax highlights
for LUA, but any other text editor is OK) open \texttt{foo.lua} and
change the first few lines adding the plugin name, version, your name,
your email and a short comment in the proper places.\linespread{0.5}

\begin{footnotesize}

\begin{verbatim}

-- ************************************************************************** -- 

--  FreePOPs @--put here domain-- webmail interface 

--  

--  $Id$ 

--  

--  Released under the GNU/GPL license 

--  Written by --put Name here-- <--put email here--> 

-- ************************************************************************** --



PLUGIN_VERSION = "--put version here--" 

PLUGIN_NAME = "--put name here--" 

\end{verbatim}

\end{footnotesize} Now we have an empty plugin, but it is not enough to start hacking
on it. We need to open the \texttt{config.lua} file (in the win32
distribution it is placed in the main directory, while in the Unix
distribution it is in \texttt{/etc/freepops/}; other copies of this
file may be included in the distributions, but they are backup copies)
and add a line like this\linespread{0.5}

\begin{footnotesize}

\begin{verbatim}

-- foo plugin 

freepops.MODULES_MAP["foo.xx"]      = {name="foo.lua"} 

\end{verbatim}

\end{footnotesize} at the beginning of the file. Before ending the first step you should
try if the plugin is correctly activated by FreePOPs when needed.
To do this we have to add few lines to \texttt{foo.lua}, in particular
we have to add an error return value to \texttt{user()}.\linespread{0.5}

\begin{footnotesize}

\begin{verbatim}

-- -------------------------------------------------------------------------- -- 

-- Must save the mailbox name 

function user(pstate,username)   

        return POPSERVER_ERR_AUTH 

end 

\end{verbatim}

\end{footnotesize} Now the user function always fails, returning an authentication error.
Now you have to start FreePOPs (if it is already running you don't
have to restart it) and start telnet (under win32 you should open
a DOS prompt, under Unix you have the shell) and type \texttt{telnet
localhost 2000} and then type \texttt{user test@foo.xx}.\linespread{0.5}

\begin{footnotesize}

\begin{verbatim}

tassi@garfield:~$ telnet localhost 2000 

Trying 127.0.0.1... 

Connected to garfield. 

Escape character is '^]'. 

+OK FreePOPs/0.0.10 pop3 server ready 

user test@foo.xx 

-ERR AUTH FAILED 

Connection closed by foreign host. 

\end{verbatim}

\end{footnotesize} The server responds closing the connection and printing an authorization
failed message (thats OK, since the \texttt{user()} function of our
plugin returns this error). In the standard error file (the console
under Unix, the file \texttt{stderr.txt} under Windows) the error
messages get printed, don't mind them now.


\subsubsection{(step 2) The login}

The login procedure is the first thing we have to do. The POP3 protocol
has 2 commands for login, \emph{user} and \emph{pass}. First the client
does a user, then it tells the server the password. As we have already
seen in the overview this means that first \texttt{user()} and then
\texttt{\emph{}}\texttt{pass()} will be called. This is a sample login:\linespread{0.5}

\begin{footnotesize}

\begin{verbatim}

tassi@garfield:~$ telnet localhost 2000 

Trying 127.0.0.1... 

Connected to garfield. 

Escape character is '^]'. 

+OK FreePOPs/0.0.10 pop3 server ready 

user test@foo.xx 

+OK PLEASE ENTER PASSWORD 

pass hello 

-ERR AUTH FAILED 

\end{verbatim}

\end{footnotesize} If you start FreePOPs with the \texttt{-w} switch you should read
this on standard error/standard output:\linespread{0.5}

\begin{footnotesize}

\begin{verbatim}

freepops started with loglevel 2 on a little endian machine. 

Cannot create pid file "/var/run/freepopsd.pid" 

DBG(popserver.c, 162): [5118] ?? Ip address 0.0.0.0 real port 2000

DBG(popserver.c, 162): [5118] ?? Ip address 127.0.0.1 real port 2000

DBG(popserver.c, 162): [5118] -> +OK FreePOPs/0.0.10 pop3 server ready

DBG(popserver.c, 162): [5118] <- user test@foo.xx

DBG(log_lua.c,  83): (@src/lua/foo.lua, 37) : FreePOPs plugin 'Foo web mail' version '0.0.1' started!

*** the user wants to login as 'test@foo.xx' 

DBG(popserver.c, 162): [5118] -> +OK PLEASE ENTER PASSWORD

DBG(popserver.c, 157): [5118] <- PASS ********* 

*** the user inserted 'hello' as the password for 'test@foo.xx' 

DBG(popserver.c, 162): [5118] -> -ERR AUTH FAILED

AUTH FAILED 

DBG(threads.c,  81): thread 0 will die 

\end{verbatim}

\end{footnotesize}and the plugin has been changed a bit to store the user login and
print some debug info. This is the plugin that gave this output:\linespread{0.5}

\begin{footnotesize}

\begin{verbatim}

foo_globals= {  

       username="nothing",

       password="nothing" 

} 

-- -------------------------------------------------------------------------- -- 

-- Must save the mailbox name 

function user(pstate,username)   

        foo_globals.username = username

        print("*** the user wants to login as '"..username.."'")

        return POPSERVER_ERR_OK

end 

-- -------------------------------------------------------------------------- -- 

-- Must login 

function pass(pstate,password)

        foo_globals.password = password

        print("*** the user inserted '"..password..

            "' as the password for '"..foo_globals.username.."'")

        return POPSERVER_ERR_AUTH end 

-- -------------------------------------------------------------------------- -- 

-- Must quit without updating 

function quit(pstate)      

        return POPSERVER_ERR_OK 

end 

\end{verbatim}

\end{footnotesize}Here we have some important news. First the \texttt{foo\_globals}
table that will contain all the globals (values that should be available
to successive function calls) we need. So far we have the username
and password there. The \texttt{user()} function now stores the passed
username in the \texttt{foo\_globals} table and prints something on
standard output. The \texttt{pass()} function likewise stores the
password in the global table and prints some stuff. The \texttt{quit()}
function simply returns \texttt{POPSERVER\_ERR\_OK} to make FreePOPs
happy.

Now that we know how FreePOPs will act during the login we have to
implement the login in the webmail, but first uncomment the few lines
in the \texttt{init()} function (that is called when the plugin is
started), that loads the \texttt{browser.lua} module (the module we
will use to login in the webmail). Here is the webmail login page
viewed with Mozilla and the source of the page (you can see it with
Mozilla with Ctrl-U, \imageref{login}).

\myincludegraphics{login}{!htp}{scale=0.8}

\linespread{0.5}

\begin{footnotesize}

\begin{verbatim}

<html> 

<head> 

<title>foo.xx webmail login</title> 

</head> 

<body style="background-color : grey; color : white"> 

<h1>Webmail login</h1>

<form name="webmail" method="post" action="http://localhost:3000/"> 

login: <input type="text" size="10" name="username"> <br> 

password: <input type="password" size="10" name="password"> <br> 

<input type="submit" value="login"> 

</form> 

</body> 

</html>

\end{verbatim}

\end{footnotesize}Here we have 2 input fields, one called username and one called password.
When the user clicks login the web browser will \texttt{POST} to \texttt{HTTP://localhost:3000/}
the form contents (I used a local address for comfort, but it should
be something like \texttt{HTTP://webmail.foo.xx/login.php}). This
is what the browser sends:\linespread{0.5}

\begin{footnotesize}

\begin{verbatim}

POST / HTTP/1.1 

Host: localhost:3000 

User-Agent: Mozilla/5.0 (X11; U; Linux i686; en-US; rv:1.6) Gecko/20040614 Firefox/0.8 Accept: */*

Accept-Language: en-us,en;q=0.5 

Accept-Encoding: gzip,deflate 

Accept-Charset: ISO-8859-1,utf-8;q=0.7,*;q=0.7 

Keep-Alive: 300 

Connection: keep-alive 

Content-Type: application/x-www-form-urlencoded 

Content-Length: 37



username=test%40foo.xx&password=hello 

\end{verbatim}

\end{footnotesize} We are not interested in the first part (the HTTP header, since the
browser module will take care of it) but in the last part, the posted
data. Since the fields of the form were username and password, the
posted data is\texttt{}~\\
\texttt{username=test\%40foo.xx\&password=hello}. Now we want to reproduce
the same HTTP request with our plugin. This is the simple code that
will do just that.\linespread{0.5}

\begin{footnotesize}

\begin{verbatim}

-- -------------------------------------------------------------------------- -- 

-- Must login 

function pass(pstate,password)

       foo_globals.password = password

        

       print("*** the user inserted '"..password..

            "' as the password for '"..foo_globals.username.."'") 



       -- create a new browser

       local b = browser.new()



       -- store the browser object in globals  

       foo_globals.browser = b



       -- create the data to post      

       local post_data = string.format("username=%s&password=%s",

               foo_globals.username,foo_globals.password)

       -- the uri to post to   

       local post_uri = "http://localhost:3000/"



       -- post it      

       local file,err = nil, nil       

  

       file,err = b:post_uri(post_uri,post_data)

       

       print("we received this webpage: ".. file)      

       return POPSERVER_ERR_AUTH 

end 

\end{verbatim}

\end{footnotesize} First we create a browser object, then we build the \texttt{post\_uri}
and \texttt{post\_data} using a simple \texttt{string.format} (printf-like
function). And this is the resulting request\linespread{0.5}

\begin{footnotesize}

\begin{verbatim}

POST / HTTP/1.1 

User-Agent: Mozilla/5.0 (X11; U; Linux i686; en-US; rv:1.6) Gecko/20040322 Firefox/0.8

Pragma: no-cache 

Accept: */* 

Host: localhost 

Content-Length: 35 

Content-Type: application/x-www-form-urlencoded



username=test@foo.xx&password=hello 

\end{verbatim}

\end{footnotesize}that is essentially the same (we should url-encode the post data with
\texttt{curl.escape()}) we wanted to do. We saved the browser object
to the global table, since we want to use the same browser all the
time.

Now that we have logged in, we want to check the resulting page, and
maybe extract a session ID that will be used later. This is the code
to extract the session id and the HTML page we have received in response
to the login request\linespread{0.5}

\begin{footnotesize}

\begin{verbatim} 

        ... the same as before here ...

     

        print("we received this webpage: ".. file)

        

        -- search the session ID        

        local _,_,id = string.find(file,"session_id=(%w+)")



        if id == nil then               

               return POPSERVER_ERR_AUTH

        end



        foo_globals.session_id = id

        return POPSERVER_ERR_OK

end 

\end{verbatim}

\end{footnotesize} and \imageref{logindone} is the returned web page.\\
\myincludegraphics{logindone}{!htp}{scale=0.8}\linespread{0.5}

\begin{footnotesize}

\begin{verbatim}

<html> 

<head> 

<title>foo.xx webmail</title> 

</head> 

<body style="background-color : grey; color : white"> 

<h1>Webmail - test@foo.xx</h1> 

Login done! click here to view the inbox folder. 

<a href="http://localhost:3000/inbox.php?session_id=ABCD1234">inbox</a> 

</body> 

</html>

\end{verbatim}

\end{footnotesize} Note that we extracted the session ID string using \\
\texttt{string.find(file,''session\_id=(\%w+)'')}. This is a really
important function in the lua library and, even if it is described
in the lua tutorial at HTTP://lua-users.org, we will talk a bit about
captures here. Look at the page source. We are interested in the line
\emph{}\\
\texttt{<a href=\char`\"{}HTTP://localhost:3000/inbox.php?session\_id=ABCD1234\char`\"{}>inbox</a>}
\emph{}\\
that contains the session\_id we want to capture. Our expression is
\texttt{\emph{session\_id=(\%w+)}} that means we want to match all
the strings that start with \texttt{session\_id=} and than continue
with one or more alphanumerical character. Since we wrote \texttt{\%w+}
in round brackets, we mean to capture the content of brackets (the
alphanumerical part). So string.find will return 3 values, the first
two are ignored (assigned to the dummy variable \texttt{\_}) while
the third is the captured string (in our case \texttt{ABCD1234}).
The LUA tutorial at lua-users is quite good and at HTTP://sf.net/projects/lua-users
you can find the LUA short reference that is a summary of all standard
lua functions and is a really good piece of paper (so many thanks
to Enrico Colombini). If you really like LUA you should buy THE book
about LUA called \emph{{}``Programming in Lua''} by Roberto Ierusalimschy
(consider it the K\&R for LUA).


\subsubsection{(step 3) Getting the list of messages}

Now we have to implement the \texttt{stat()} function. The stat is
probably the most important function. It must retrieve the list of
messages in the webmail and their UIDL and size. In our example we
will use the mlex module to grab the important info from the page,
but you can use the string LUA module to do the same with captures.
This is our inbox page (see \imageref{inbox})\\
\myincludegraphics{inbox}{!htp}{scale=0.8}

and this is the HTML body (only the first 2 messages are reported)\linespread{0.5}

\begin{footnotesize}

\begin{verbatim}

<h1>test@foo.xx - inbox (1/2)</h1> 

<form name="inbox" method="post" action="/delete.php"> 

<input type="hidden" name="session_id" value="ABCD1234"> 

<table> 

<tr><th>From</th><th>subject</th><th>size</th><th>date</th></tr> 

<tr>        

  <td><b>friend1@foo1.xx</b></td>         

  <td><b><a href="/read.php?session_id=ABCD1234&uidl=123">ok!</a></b></td>

  <td><b>20KB</b></td>

  <td><b>today</b></td>   

  <td><input type="checkbox" name="check_123"></td>

</tr> 

<tr>    

  <td>friend2@foo2.xx</td>        

  <td><a href="/read.php?session_id=ABCD1234&uidl=124">Re: hi!</a></td>  

  <td>12KB</td>   

  <td>yesterday</td>      

  <td><input type="checkbox" name="check_124"></td> 

</tr>

</table> 

<input type="submit" value="delete marked"> 

</form> 

<a href="/inbox.php?session_id=ABCD1234&page=2">go to next page</a> 

</body> 

\end{verbatim}

\end{footnotesize}We have retrieved the HTML using the browser and the \texttt{get\_uri()}
method (remember the URI for the inbox was in the login page). As
you can see the messages are in a table and this table has the same
structure for each message. This is the place in which you may use
mlex. Just take all the stuff between \texttt{<tr>} and \texttt{</tr>}
of a message row and delete all but the tags name. Then replace every
empty space (we call space the string between two tags) with a {}``\texttt{.{*}}''.
This is what we have obtained (it should be all in the same line,
here is wrapped for lack of space) from the first message.\linespread{0.5}

\begin{footnotesize}

\begin{verbatim}

.*<tr>.*<td>.*<b>.*</b>.*</td>.*<td>.*<b>.*<a>.*</a>.*</b>.*</td>.*

<td>.*<b>.*</b>.*</td>.*<td>.*<b>.*</b>.*</td>.*

<td>.*<input>.*</td>.*</tr>

\end{verbatim}

\end{footnotesize}This expression is used to match the table row containing info about
the message. Now cut and paste the line and replace every space and
every tag with O (the letter, not the digit 0) or X. Put an X in the
interesting fields (in our example the size and the input tag, that
contains the message uidl).\linespread{0.5}

\begin{footnotesize}

\begin{verbatim}

O<O>O<O>O<O>O<O>O<O>O<O>O<O>O<O>O<O>O<O>O<O>O

<O>O<O>X<O>O<O>O<O>O<O>O<O>O<O>O

<O>O<X>O<O>O<O>

\end{verbatim}

\end{footnotesize}While the first expression will be used to match the table row, this
one will be used to extract the important fields. This is the code
that starts mlex on the HTML and fills the popstate data structure
with the captured data.\linespread{0.5}

\begin{footnotesize}

\begin{verbatim}

-- -------------------------------------------------------------------------- -- 

-- Fill the number of messages and their size 

function stat(pstate)

      local file,err = nil, nil

      local b = foo_globals.browser

      file,err = b:get_uri("http://localhost:3000/inbox.php?session_id="..

              foo_globals.session_id)

      local e = ".*<tr>.*<td>.*<b>.*</b>.*</td>.*<td>.*<b>.*<a>"..               

              ".*</a>.*</b>.*</td>.*<td>.*<b>.*</b>.*</td>.*<td>.*"..                

              "<b>.*</b>.*</td>.*<td>.*<input>.*</td>.*</tr>"         

      local g = "O<O>O<O>O<O>O<O>O<O>O<O>O<O>O<O>O<O>O<O>O<O>O"..              

              "<O>O<O>X<O>O<O>O<O>O<O>O<O>O<O>O<O>O<X>O<O>O<O>"

      local x = mlex.match(file,e,g) 

      --debug print   

      x:print()



      set_popstate_nummesg(pstate,x:count())

      for i=1,x:count() do            

              local _,_,size = string.find(x:get(0,i-1),"(%d+)")

              local _,_,size_mult_k = string.find(x:get(0,i-1),"([Kk][Bb])")                        local _,_,size_mult_m = string.find(x:get(0,i-1),"([Mm][Bb])")

              local _,_,uidl = string.find(x:get(1,i-1),"check_(%d+)")

           

              if size_mult_k ~= nil then

                     size = size * 1024

              end             

              if size_mult_m ~= nil then

                     size = size * 1024 * 1024

              end             

     

              set_mailmessage_size(pstate,i,size)                     

              set_mailmessage_uidl(pstate,i,uidl)

      end

 

      return POPSERVER_ERR_OK

end 

\end{verbatim}

\end{footnotesize}The result of \texttt{x:print()} is the following\linespread{0.5}

\begin{footnotesize}

\begin{verbatim}

{'20KB','input type="checkbox" name="check_123"'}

\end{verbatim}

\end{footnotesize}and the telnet session follows\linespread{0.5}

\begin{footnotesize}

\begin{verbatim}

+OK FreePOPs/0.0.11 pop3 server ready 

user test@foo.xx 

+OK PLEASE ENTER PASSWORD 

pass secret 

+OK ACCESS ALLOWED 

stat 

+OK 1 20480 

quit 

+OK BYE BYE, UPDATING 

\end{verbatim}

\end{footnotesize}We have not listed here how we added the dummy \texttt{return} \texttt{POPSERVER\_ERR\_OK}
line to the \texttt{quit()} function. The source code listed before
uses mlex to extract the two interesting strings, then parses them
searching for the size and the size multiplier and the uidl. Then
sets the mail message attributes. But here you can see that we just
matched the first message. To match the other messages we have to
inform the mlex module that the \texttt{<b>} tag is optional (you
can see that only the first message is in bold). So we change the
expressions to\linespread{0.5}

\begin{footnotesize}

\begin{verbatim}

.*<tr>.*<td>[.*]{b}.*{/b}[.*]</td>.*<td>[.*]{b}.*<a>.*</a>.*{/b}[.*]</td>.*

<td>[.*]{b}.*{/b}[.*]</td>.*<td>[.*]{b}.*{/b}[.*]</td>.*

<td>.*<input>.*</td>.*</tr>

\end{verbatim}

\end{footnotesize}and\linespread{0.5}

\begin{footnotesize}

\begin{verbatim}

O<O>O<O>[O]{O}O{O}[O]<O>O<O>[O]{O}O<O>O<O>O{O}[O]<O>O

<O>[O]{O}X{O}[O]<O>O<O>[O]{O}O{O}[O]<O>O

<O>O<X>O<O>O<O>

\end{verbatim}

\end{footnotesize}Now the stat command responds with \texttt{+OK 4 45056} and the debug
print is \linespread{0.5}

\begin{footnotesize}

\begin{verbatim}

{'20KB','input type="checkbox" name="check_123"'} 

{'12KB','input type="checkbox" name="check_124"'} 

{'10KB','input type="checkbox" name="check_125"'} 

{'2KB','input type="checkbox" name="check_126"'}

\end{verbatim}

\end{footnotesize}Now we have a proper function stat that fill the popstate data structure
with the info the popserver needs to respond to a stat request. Since
the list, uidl, list\_all and uidl\_all requests can be satisfied
with the same data we will use the standard function provided by the
common.lua module. It will be explained in the next step, but we have
to add 2 important lines to the  \texttt{stat()} function, to avoid
a double call.\linespread{0.5}

\begin{footnotesize}

\begin{verbatim}

function stat(pstate) 

       if foo_globals.stat_done == true then return POPSERVER_ERR_OK end



       ... the same code here ...



       foo_globals.stat_done = true

       return POPSERVER_ERR_OK

end

\end{verbatim}

\end{footnotesize}

The most important function is done, but a lot of notes must be written
here. First, mlex is really comfortable sometimes, but you may find
more helpful using the lua string library or the regularexp library
(posix extended regular expressions) to reach the same point. Second,
this implementation stops at the first inbox page. You should visit
all the inbox pages maybe using the \texttt{do\_until()} function
in the \texttt{support.lua} library (that will be briefly described
at the end of this tutorial). Third we make no error checking. For
example the file variable may be nil and we must check these things
to make a good plugin.


\subsubsection{(step 4) The common functions}

The common module gives us some pre-cooked functions that depend only
on a well implemented \texttt{stat()} (I mean a stat than can be called
more than once). This is our implementation of these functions \linespread{0.5}

\begin{footnotesize}

\begin{verbatim}

-- -------------------------------------------------------------------------- -- 

-- Fill msg uidl field 

function uidl(pstate,msg) return common.uidl(pstate,msg) end 



-- -------------------------------------------------------------------------- -- 

-- Fill all messages uidl field 

function uidl_all(pstate) return common.uidl_all(pstate) end 



-- -------------------------------------------------------------------------- -- 

-- Fill msg size 

function list(pstate,msg) return common.list(pstate,msg) end 



-- -------------------------------------------------------------------------- -- 

-- Fill all messages size 

function list_all(pstate) return common.list_all(pstate) end 



-- -------------------------------------------------------------------------- -- 

-- Unflag each message marked for deletion 

function rset(pstate) return common.rset(pstate) end



-- -------------------------------------------------------------------------- -- 

-- Mark msg for deletion 

function dele(pstate,msg) return common.dele(pstate,msg) end 



-- -------------------------------------------------------------------------- -- 

-- Do nothing 

function noop(pstate) return common.noop(pstate) end 



\end{verbatim}

\end{footnotesize}but first add the common module loading code to your \texttt{init()}
function\linespread{0.5}

\begin{footnotesize}

\begin{verbatim}

        ... the same code ..



        -- the common module    

        if freepops.dofile("common.lua") == nil then

                  return POPSERVER_ERR_UNKNOWN    

        end 



        -- checks on globals    

        freepops.set_sanity_checks()    



        return POPSERVER_ERR_OK 

end 

\end{verbatim}

\end{footnotesize}


\subsubsection{(step 5) Deleting messages}

Deleting messages is usually a normal post and an example of the \texttt{post\_data}
is \texttt{session\_id=ABCD1234\&check\_124=on\&check\_126=on}. The
code follows\linespread{0.5}

\begin{footnotesize}

\begin{verbatim}

-- -------------------------------------------------------------------------- -- 

-- Update the mailbox status and quit 

function quit_update(pstate)

      -- we need the stat

      local st = stat(pstate)

      if st ~= POPSERVER_ERR_OK then return st end

         

      -- shorten names, not really important  

      local b = foo_globals.b         

      local post_uri = b:wherearewe() .. "/delete.php"        

      local session_id = foo_globals.session_id       

      local post_data = "session_id=" .. session_id .. "&"

    

      -- here we need the stat, we build the uri and we check if we   

      -- need to delete something     

    

      local delete_something = false; 

      for i=1,get_popstate_nummesg(pstate) do                 

             if get_mailmessage_flag(pstate,i,MAILMESSAGE_DELETE) then                                            post_data = post_data .. "check_" .. 

                        get_mailmessage_uidl(pstate,i).. "=on&"                             

                    delete_something = true                 

             end     

      end

    

      if delete_something then                

             b:post_uri(post_uri,post_data)  

      end

    

      return POPSERVER_ERR_OK 

end 

\end{verbatim}

\end{footnotesize}Consider we do the post only if at least one message is marked for
deletion. Another important think to keep in mind is that making only
one post for all messages is better than making a single post for
each message. When it is possible you should reduce the number of
HTTP requests as much as you can since it is here we move FreePOPs
from a rabbit to a tortoise.


\subsubsection{(step 6) Downloading messages}

You may ask why I talk about this only at point 6, while having the
mail is probably what you want from a plugin. Implementing the \texttt{retr()}
function is usually simple. It really depends on the webmail, but
here we will talk of the simple case, while at the end of the tutorial
you will see how to deal with complex webmails. The simple case is
the one in which the webmail has a save message button. And the saved
message is a plain text file containing both the header and the body
of the message. There are only two interesting points in this case,
firstly big messages, secondly the dot issue.

Big messages are a cause of timeout. Yes, the most simple way of downloading
a message is calling \texttt{b:get\_uri()} and store the message in
a variable, and then send it to the mail client with \texttt{popserver\_callback()}.
But think that a 5MB mail, downloaded with a 640Kbps DSL connection,
at full 80KBps speed, takes 64 seconds to download. This means your
plugin will not send any data to the mail client for more than one
minute and this will make the mail client to disconnect from FreePOPs
 thinking the POP3 server is dead. So we must send the data to the
mail client as soon as we can. For this we have the \texttt{b:pipe\_uri()}
function that calls a callback whenever it has some fresh data. The
following code is the callback factory function, that creates a new
callback to pass to the \texttt{pipe\_uri} browser method.\linespread{0.5}

\begin{footnotesize}

\begin{verbatim}

-------------------------------------------------------------------------------- 

-- The callback factory for retr 

-- 

function retr_cb(data)

        local a = stringhack.new() 

        return function(s,len)

                s = a:dothack(s).."\0"          

                popserver_callback(s,data)              

                return len,nil

        end

end 

\end{verbatim}

\end{footnotesize}Here you see that the callback simply uses \texttt{popserver\_callback()}
to pass the data to the mail client, but before doing this it mangles
the data with the stringhack. But this is the second interesting point.

The POP3 protocol should end the retr command answer with a line that
contains only 3 bytes, {}``\texttt{.\textbackslash{}r\textbackslash{}n}''.
But what if a line, inside the mail body, is a simple point? We have
to escape it to {}``\texttt{..\textbackslash{}r\textbackslash{}n}''.
This is not so hard, a \texttt{string.gsub(s,''\textbackslash{}r\textbackslash{}n.\textbackslash{}r\textbackslash{}n'',''\textbackslash{}r\textbackslash{}n..\textbackslash{}r\textbackslash{}n'')}
is all we need... but not in the case of callbacks. The send callback
will be called with some fresh data, and called more than once if
the mail is big. And if the searched pattern is truncated between
two calls the \texttt{string.gsub()} method will fail. This is why
the stringhack module helps us. The \texttt{a} object lives as long
as the callback function will be called (see the closure page of the
lua tutorial) and will keep in mind that the searched pattern may
be truncated.

Finally the \texttt{retr()} code\linespread{0.5}

\begin{footnotesize}

\begin{verbatim}

-- -------------------------------------------------------------------------- -- 

-- Get message msg, must call  

-- popserver_callback to send the data 

function retr(pstate,msg,pdata)          

        -- we need the stat

        local st = stat(pstate)

        if st ~= POPSERVER_ERR_OK then return st end 

    

        -- the callback

        local cb = retr_cb(data) 

     

        -- some local stuff

       local session_id = foo_globals.session_id

       local b = internal_state.b

       local uri = b:wherearewe() .. "/download.php?session_id="..session_id..

                "&message="..get_mailmessage_uidl(pstate,msg) 



        -- tell the browser to pipe the uri using cb

        local f,rc = b:pipe_uri(uri,cb)

        if not f then

                log.error_print("Asking for "..uri.."\n")

                log.error_print(rc.."\n")

                return POPSERVER_ERR_NETWORK

        end

end 

\end{verbatim}

\end{footnotesize}


\subsubsection{(step 7) Test it}

Making a good plugin needs a lot of testing. You should ask for beta
testers at the FreePOPs forum (HTTP://freepops.diludovico.it) and
ask the software authors to include it in the main distribution. You
should also read the webmail contract, check if there is something
like {}``\emph{I'll never use webmail->pop3 server to read my mail}''
and send a copy to the authors of the software.


\subsubsection{(step 8) The so mentioned last part of the tutorial}

There are a lot of things we have omitted here.

\begin{description}
\item [The~multi-page~stat]is the real good implementation for \texttt{stat()}.
We mentioned before that our implementation lists only the messages
in the first page. The code for parsing and extracting interesting
info from a page is already written, we simply need a function that
checks if we are in the last page and if not it changes the value
of a \texttt{uri} variable. The \texttt{uri} variable will be used
by the fetch function. In this case you should use the support module
with the do\_until cycle. This is a simple example of \texttt{do\_until()}
\linespread{0.5}

\begin{footnotesize}

\begin{verbatim}

-- -------------------------------------------------------------------------- -- 

-- Fill the number of messages and their size 

function stat(pstate)

        ... some code as before ...



        -- this string will contain the uri to get. it may be updated by        

        -- the check_f function, see later      

        local uri = string.format(libero_string.first,popserver,session_id)



        -- The action for do_until      

        --      

        -- uses mlex to extract all the messages uidl and size  

        local function action_f (s)

                 -- calls match on the page s, with the mlexpressions

                 -- statE and statG              

                 local x = mlex.match(s,e,g)                 

               

                 -- the number of results                

                 local n = x:count()

                

                 if n == 0 then return true,nil end

 

                 -- this is not really needed since the structure                

                 -- grows automatically... maybe... don't remember now 

                 local nmesg_old = get_popstate_nummesg(pstate)

                 local nmesg = nmesg_old + n 

                 set_popstate_nummesg(pstate,nmesg)

      

                 -- gets all the results and puts them in the popstate structure                              for i = 1,n do                        

                         ... some code as before ...

 

                         set_mailmessage_size(pstate,i+nmesg_old,size)                     

                         set_mailmessage_uidl(pstate,i+nmesg_old,uidl)           

                 end     

                

                 return true,nil         

        end

        

        -- check must control if we are not in the last page and        

        -- eventually change uri to tell retrieve_f the next page to retrieve     

        local function check_f (s)              

                 local tmp1,tmp2 = string.find(s,next_check)              

                 if tmp1 ~= nil then                     

                          -- change retrieve behavior                     

                          uri = "--build the uri for the next page--"



                          -- continue the loop

                          return false       

                 else

                          return true

                 end

        end



        -- this is simple and uri-dependent

        local function retrieve_f ()

                 local f,err = b:get_uri(uri)

                 if f == nil then 

                         return f,err

                 end

      

                 local _,_,c = string.find(f,"--timeout string--")

                 if c ~= nil then

                         internal_state.login_done = nil                                

                         session.remove(key())

                         local rc = libero_login()                       

                         if rc ~= POPSERVER_ERR_OK then                          

                                 return nil,"Session ended,unable to recover"                                         end             

                        

                         uri = "--uri for the first page--"      

                         return b:get_uri(uri)           

                  end     

               

                  return f,err    

        end



        -- initialize the data structure

        set_popstate_nummesg(pstate,0)

 

        -- do it        

        if not support.do_until(retrieve_f,check_f,action_f) then

                  log.error_print("Stat failed\n")

                  session.remove(key())           

                  return POPSERVER_ERR_UNKNOWN    

        end

        

        -- save the computed values     

        internal_state["stat_done"] = true 

        return POPSERVER_ERR_OK 

end 

\end{verbatim}

\end{footnotesize}The only strange things are the retrieve function and the session
saving stuff. Since webmail sometimes timeout you should check if
the retrieved page is valid or not, and eventually retry the login.
The session saving is the next issue.
\item [Saving~the~session]is the way to make FreePOPs really similar
to a browser. This means the next time you check the mail FreePOPs
will simply reload the inbox page and won't login again. To do this
you need a \texttt{key()} function that gives a unique ID for each
session\linespread{0.5}

\begin{footnotesize}

\begin{verbatim}

-------------------------------------------------------------------------------- 

-- The key used to store session info 

-- 

-- This key must be unique for all webmails, since the session pool is one  

-- for all the webmails 

-- 

function key()

        return foo_globals.username .. foo_globals.password

end 

\end{verbatim}

\end{footnotesize}and a \texttt{foo\_globals} serialization function\linespread{0.5}

\begin{footnotesize}

\begin{verbatim}

-------------------------------------------------------------------------------- 

-- Serialize the internal state 

-- 

-- serial.serialize is not enough powerful to correctly serialize the  

-- internal state. The field b is the problem. b is an object. This means 

-- that it is a table (and no problem for this) that has some field that are 

-- pointers to functions. this is the problem. there is no easy way for the  

-- serial module to know how to serialize this. so we call b:serialize  

-- method by hand hacking a bit on names 

-- 

function serialize_state()   

        internal_state.stat_done = false; 

        return serial.serialize("foo_globals",foo_globals) ..            

                internal_state.b:serialize("foo_globals.b") 

end 

\end{verbatim}

\end{footnotesize}Now you have to tell FreePOPs to save the state in the \texttt{quit\_update()}
function and load it back in the \texttt{pass()} one. This is the
new \texttt{pass()} structure\linespread{0.5}

\begin{footnotesize}

\begin{verbatim}

function pass(pstate,password)  

        -- save the password

        internal_state.password = password



        -- eventually load session

        local s = session.load_lock(key())



        -- check if loaded properly

        if s ~= nil then

                 -- "\a" means locked

                 if s == "\a" then

                          log.say("Session for "..internal_state.name..

                              " is already locked\n")

                          return POPSERVER_ERR_LOCKED

                 end 

     

                 -- load the session

                 local c,err = loadstring(s)

                 if not c then

                          log.error_print("Unable to load saved session: "..err)

                          return foo_login()

                 end     

       

                 -- exec the code loaded from the session string

                 c()



                log.say("Session loaded for " .. internal_state.name .. "@" ..

                         internal_state.domain ..

                         "(" .. internal_state.session_id .. ")\n")      



                return POPSERVER_ERR_OK

        else

                -- call the login procedure

                return foo_login()

        end

end 

\end{verbatim}

\end{footnotesize}where \texttt{foo\_login()} is the old \texttt{pass()} function with
minor changes. Don't forget to call \texttt{session.unlock(key())}
in the \texttt{quit()} function, since you have to release the session
in case of failure (and \texttt{quit()} is called here) and to save
the session in \texttt{quit\_update()}\linespread{0.5}

\begin{footnotesize}

\begin{verbatim}

-- save fails if it is already saved  

session.save(key(),serialize_state(),session.OVERWRITE)

-- unlock is useless if it have just been saved, but if we save

-- without overwriting the session must be unlocked manually

-- since it would fail instead overwriting

session.unlock(key()) 

\end{verbatim}

\end{footnotesize}
\item [The~top()~function]is a complex thing. I won't describe it in
a complete way, but I suggest you to look at the \texttt{libero.lua}
plugin if the web server that sends you the message source supports
the {}``\texttt{Range:}'' HTTP request field, or the \texttt{tin.lua}
plugin if the server needs to be interrupted in a bad way. Remember
that the \texttt{top()} needs someone that counts the lines and here
we have again the stringhack module that counts and may purge some
lines.
\item [The~javascript]is the hell of webmails. Javascripts can do anything
and you have to read them to emulate what they do. For example they
may add some cookies (and you'll have to do this by hand with the
\texttt{b:add\_cookie()} as in tin.lua) or they may change some form
fields (like in the \texttt{libero.lua} login load balancing code).
\item [The~cookies]are sweet enough for us, since the browser module will
handle them for us.
\item [The~standard~files]are really system dependent. Under Windows
you'll have to constantly look at the \texttt{stderr.txt} and \texttt{stdout.txt},
while under Unix you will just have to start it with the \texttt{-w}
switch and look at the console.
\item [The~brute~force]is called Ethereal. Sometimes things don't work
in the right way and the only way to debug them is to activate curl
debugging to see what FreePOPs does (\texttt{b.curl:setopt(curl.OPT\_VERBOSE,1)})
and sniff what a real browser does with a tool like Ethereal.
\item [The~open~source~way]is the best way of having a good quality
piece of software. This means you'll have to release really often
your plugin in the development phase and interact much with your testers.
Trust me it works, or read {}``\emph{The cathedral and the bazaar}''
by Eric Raymond.
\item [The~mimer~module]is really beta at the time of this tutorial,
but is what you need if you are in the unlucky case of a webmail that
has no save message button. The \texttt{lycos.lua} plugin is an example
of what it can do. The main interesting function is \texttt{mimer.pipe\_msg()}
that takes a message header, a text body (in html o plain text format)
and some attachments URIs, that are downloaded on the fly, composed
into a proper mail message and piped to the mail client.
\item [Parameters~to~modules]may be passed from the \texttt{config.lua}
file or on the fly using \texttt{user@domain?param1=value1\&...\&paramN=valueN}
as described in the plugins chapter. For the plugin writer there is
no difference between the two passing mechanisms. The parameters are
available to the plugin in the table \texttt{freepops.MODULE\_ARGS}.
\item [Regex~to~define~handled~domains]are allowed since version
0.0.29. Official plugins can have a config.lua line like
\begin{verbatim}
freepops.MODULES_MAP["foo2.*"]  = {

name="foo.lua",

regex = true -- enables the regex processing

}
\end{verbatim}
while unofficial plugins can declare a regexp list in the 
\texttt{PLUGIN\_REGEXES} field. For example
\begin{verbatim}
PLUGIN_REGEXES = {"@foo2.*", "@foo3.*", "@foo4[A-Z]*"}
\end{verbatim}
\end{description}

\section{Submitting a bug}

When you have problems or you think you have found a bug, please follow
strictly this \emph{iter}:

\begin{enumerate}
\item Update to the most recent version of FreePOPs.
\item Try to reproduce the bug, if the bug is not easily reproducible we
are out of luck. Something can still be tried: if the software crashed
you could compile it from the sources, install valgrind, run freepopsd
with valgrind and hope the error messages are interesting.
\item Clean the log files
\item Start FreePOPs with the -w switch
\item Reproduce the bug
\item Send to the developers the log, plus any other info like your system
type and how to reproduce this bug.
\end{enumerate}

\section{FAQ}


\subsection*{How do I configure FreePOPs?}

In normal use conditions you don't need to configure FreePOPs, you
just have to change your mail client settings as described in the
tutorial. For other use cases we provided specific tutorials. Remember
to set the POP3 server address to \emph{localhost} (if you've installed
FreePOPs on the same computer where you read your mail, otherwise
use the IP address of the computer where FreePOPs is installed), the
server port to \emph{2000} (or whatever you chose if you ran FreePOPs
with the \texttt{-p} option) and to set the username to your full
email address (in the form \texttt{username@webmail.domain}).

If you continue having problems after reading the manual and the tutorial
you can post on the forum at HTTP://freepops.diludovico.it (usually
in Italian, but feel free to post in English).


\subsection*{How much does it cost?}

FreePOPs is Free Software, Free as in Freedom \emph{and} {}``Free
as in Free Beer''. You can download it, use it, and modify it freely
but if you care about the four friends who made it you may send them
a beer asking for their address via email, or a small donation (through
the project website on SourceForge).


\subsection*{I've installed FreePOPs and properly configured my mail client, but
I'm unable to send mail...?}

FreePOPs helps you only in receiving messages. To send mail you have
to use the SMTP server of your network provider. If you don't know
what SMTP to use, see your ISP website or call their tech support.


\subsection*{This software is in beta stage.. will I lose my mail?}

Nobody guarantees the software they write, even if you pay it a lot.
It looks safe enough to us, but we can't guarantee anything (no software
is really safe and secure). Besides no one guarantees your mail client
works well, so if you're using that...


\subsection*{How can I help the project?}

Use the source Luke... Sources are freely available, feel free to
send us patches and bug reports.

If you want, you can donate to the project via Sourceforge and Paypal.

When you want to report a bug, check that you are using the latest
version (we advise you uninstall an old version before installing
a new one) and don't forget to include:

\begin{itemize}
\item The version number of your copy of FreePOPs
\item The operating system you are running
\item Your mail client name and its version number
\item Most importantly the log of FreePOPs, generated with the \textbf{-w}
option (read further if you don't know how to add command line parameters).
Make sure that the log doesn't contain any sensitive info you wouldn't
like to disclose (to the authors). But don't worry, your password
is never recorded in the log, it's substituted by a sequence of nine
(9) asterisks, no matter what is its real length.
\end{itemize}

\subsection*{Where is the log?}

It depends on the system you are using. On a GNU/Linux system it is
probably in \emph{/var/log/freepops} or wherever you specified with
the \texttt{-l} option. On a Windows system the file log.txt is in
the same folder of the FreePOPs executable or wherever you specified
with the \texttt{-l} option.

Before sending the log make sure that it was generated with the \textbf{-vv}
or \textbf{-w} option (see below). These parameters generate a {}``verbose
log'' that makes it much easier for us to discover problems. We suggest
you erase the log file, start FreePOPs with the switch and recreate
the problem. Also make sure that it doesn't contain sensitive info
you wouldn't like to disclose.


\subsection*{How do I change FreePOPs's command line switches?}

On a Unix system you should know how to do this, just add the switches
to the command you use to run FreePOPs (maybe from a script).

On a Windows system you have two options: you may open the properties
of the FreePOPs link in the Start -> Programs -> FreePOPs menu (right
click on the link then select {}``Properties...'') and add the switches
in the command line in the {}``Destination'' box there (after X:\textbackslash{}Something\textbackslash{}freepopsd.exe);
or, run FreePOPs manually from a DOS window with the options you like
(again, after the name of the executable).

The command line parameters you need depend on your specific needs.
Read the appropriate section of the manual to know what all available
parameters are, or run \texttt{freepopsd -h} (also \texttt{man freepopsd}
on Unix).

\begin{itemize}
\item Unix example: \texttt{/usr/bin/freepopsd -P proxy:port -A user:pass
-w -l logfile.txt}
\item Windows example: \texttt{C:\textbackslash{}Programs\textbackslash{}FreePOPs\textbackslash{}freepopsd.exe
-w}
\end{itemize}

\subsection*{My {}``AntiVirus'' says FreePOPs is a virus!}

Stop using that crazy antivirus :-) Seriously, FreePOPs does not contain
viruses, trojan horses, worms, arcane formulae for daemonic rites,
secret plans for taking over the world or anything like all that.
If you don't believe us, the source code is available to everyone
(the program is released under the GNU GPL license) and thinking you
could hide malicious code in plain sunlight would be crazy. Therefore
any antivirus software that detects this kind of problems has accuracy
problems of its own, to say the least.

It sure is true that downloading a pre-compiled binary from an unknown
source is like asking for trouble. Pre-compiled binary packages you
find on the official website (http://www.freepops.org) come from the
source code that everyone can get. What you find on free (as in {}``free
beer'') software websites or you download through peer-to-peer file
sharing systems can't obviously guarantee any kind of safety. So use
some common sense.


\section{Authors}

This manual has been written by Enrico Tassi \texttt{<gareuselesinge [at] users.sourceforge.net>}
and revised and translated by Nicola Cocchiaro \texttt{<ncocchiaro [at] users.sourceforge.net>}


\subsection{Developers}

\noindent FreePOPs is developed by:

\begin{itemize}
\item Enrico Tassi \texttt{<gareuselesinge [at] users.sourceforge.net>}
\item Alessio Caprari \texttt{<alessiofender [at] users.sourceforge.net>}
\item Nicola Cocchiaro \texttt{<ncocchiaro [at] users.sourceforge.net>}
\item Simone Vellei \texttt{<simone\_vellei [at] users.sourceforge.net>}
\end{itemize}

\noindent yahoo.lua, hotmail.lua, aol.lua, netscape.lua, mailcom.lua,
juno.lua, mail2world.lua are developed by:

\begin{itemize}
\item Russell Schwager <\texttt{russells [at] despammed.com}> \\
\end{itemize}
\noindent gmail.lua is developed by:

\begin{itemize}
\item Rami Kattan <\texttt{rkattan [at] gmail.com}> \\
HTTP://www.kattanweb.com/rami
\end{itemize}
\noindent squirrelmail.lua, tre.lua are developed by:

\begin{itemize}
\item Eddi De Pieri <\texttt{dpeddi [at] users.sourceforge.net}> \\
\end{itemize}
\noindent supereva.lua is developed by:

\begin{itemize}
\item Andrea Dalle Molle <\texttt{Tund3r [at] fastwebnet.it}> \\
\end{itemize}

\noindent LiberoPOPs was developed by:

\begin{itemize}
\item Enrico Tassi \texttt{<gareuselesinge [at] users.sourceforge.net>}
\item Alessio Caprari \texttt{<alessiofender [at] users.sourceforge.net>}
\item Nicola Cocchiaro \texttt{<ncocchiaro [at] users.sourceforge.net>}
\item Simone Vellei \texttt{<simone\_vellei [at] users.sourceforge.net>}
\item Giacomo Tenaglia \texttt{<sonicsmith [at] users.sourceforge.net>}
\end{itemize}

\section{Thanks}

Special thanks go to the users who tested the software, to the hackers
who made it possible to have a free and reliable development environment
as the Debian GNU/Linux system.


\section*{Appendix}

Since the ASCII table is really common, we have not included one here.
You may ask google.
\end{document}
